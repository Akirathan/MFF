
\documentclass{article}

\usepackage[utf8]{inputenc}
\usepackage[czech]{babel}
\usepackage{amsmath}

\begin{document}

\title{Du 1}
\author{Pavel Marek}
\date{}

\maketitle

\section*{1)}
Za každou instrukci typu $ \delta(q_1, x) = (q_2, y, L) $ přidáme nový stav
$q_1$ a navíc tuto instrukci nahradíme dvěmi následujícími instrukcemi:
\begin{align*}
  \delta(q_1, x) &= (q_{1}', y, N)\\
  \delta(q_{1}', y) &= (q_2, y, L)
\end{align*}

\section*{2)}

Pro řešení předpokládáme, že jednopáskový, oboustraně nekonečný turingův
stroj lze převést na jednostranně nekonečný. Navíc v tomto jednostranně
nekonečném turingově stroji předpokládáme vstupní slovo oboustraně ohraničené
zarážkami.

\paragraph{Myšlenka řešení}
Ukážeme za jakou posloupnost instrukcí resp. program zaměníme každou instrukci,
která pohne hlavou doleva.

Myšlenka je taková, že vytvoříme dvě zarážky - jednu pod hlavou a druhou na
začátku vstupního slova. Levou zarážku budeme posunovat do prava, dokud mezi
sebou zarážky nebudou mít právě jedno políčko. Jakmile mezi sebou budou mít
zarážky právě jedno políčko, přepíšeme nejprve pravou zarážku na původní
znak, restartujeme hlavu, a poté přepíšeme levou zarážku na původní znak.
Tím de facto posuneme hlavu o jedno políčko doleva.

Zarážky vždy přepisujeme na původní znaky tak, abychom neměnili symboly ve
vstupním slově. Schématicky vypadá posunutí zarážky $\Phi_x$ následovně:
$$ [\Phi_x, z] \rightarrow [x, \Phi_z] $$

Poznamenejme ještě, že tímto podprogramem se dají nahradit instrukce, které
hýbou hlavou doleva. Instrukce, které hlavu nechávají na místě nahradíme
podobným podprogramem, s tím rozdílem, že levou zarážku budeme posunovat tak
dlouho, než se zarážky dostanou vedle sebe.

% Ještě si zbývá rozmyslet krají případy.

\paragraph{Detailnější popis řešení}
Každou instrukci typu $\delta(q_1, x) = (q_2, y, L)$ nahradíme následujícím
podprogramem:
\begin{enumerate}
  \item Vytvoříme zarážky:
  \begin{enumerate}
    \item Přepíšeme znak pod hlavou na $\Phi_y$
    \item Napíšeme zarážku na začátek vstupního slova.
  \end{enumerate}
  \item V cyklu:
  \begin{enumerate}
    \item Zkontrolujeme, jestli zarážky nejsou od sebe vzdálené právě 1 políčko,
    což schématicky odpovídá stavu $[\Phi_x, z, \Phi_y]$.
    \begin{enumerate}
      \item Pokud ne, tak restartujeme hlavu, a posuneme levou zarážku doprava.
      \item Pokud ano, tak:
      \begin{enumerate}
        \item Přepíšeme pravou zarážku $\Phi_y$ na $y$ a restartujeme hlavu.
        \item Přepíšeme levou zarážku $\Phi_x$ na $x$, pohneme hlavou doprava,
        a změníme stav na $q_2$. Tím jsme skončili.
      \end{enumerate}
    \end{enumerate}
  \end{enumerate}
\end{enumerate}

\end{document}
