\documentclass{article}

\usepackage[utf8]{inputenc}
\usepackage[czech]{babel}
\usepackage{amsmath}
\usepackage{amsthm}
\usepackage{amsfonts}
\usepackage{xcolor}
\usepackage{amssymb}

\newcommand{\TIME}[1]{\mathrm{TIME}(#1)}
\newcommand{\NTIME}[1]{\mathrm{NTIME}(#1)}
\newcommand{\SPACE}[1]{\mathrm{SPACE}(#1)}
\newcommand{\NSPACE}[1]{\mathrm{NSPACE}(#1)}

\author{Pavel Marek}
\title{DÚ 4}
\date{}

\begin{document}
\maketitle

\section*{Použitá tvrzení z přednášky}
Zde opíšeme a očíslujeme tvrzení z přednášek, které budeme dále používat.

\paragraph{(B)}
$\TIME{f(n)} \subseteq \NTIME{f(n)} \subseteq \SPACE{f(n)} \subseteq
\NSPACE{f(n)}$.

\paragraph{(D)}
$f(n) = o(g(n))$, $\NSPACE{f(n)} \subseteq \TIME{2^{g(n)}}$.

\paragraph{(E)}
$\NSPACE{f(n)} \subseteq \SPACE{f^2(n)}$.

\paragraph{(F)}
$f(n) = o(g(n))$, $g$ je prostorově konstruovatelná.
Potom $\SPACE{f(n)} \subsetneq \SPACE{g(n)}$.

\paragraph{(G)}
$f(n) = o(\frac{g(n)}{\log g(n)})$, $g$ je časově konstruovatelná.
Potom $\TIME{f(n)} \subsetneq \TIME{g(n)}$.


\section*{1)} 
\emph{Porovnejte $\TIME{2^n}$ a $\NSPACE{\sqrt{n}}$.}\\

Z (D) je ihned vidět, že $\NSPACE{\sqrt{n}} \subseteq \TIME{2^n}$,
protože $\sqrt{n} = o(n)$.

Pro ostrou inkluzi můžeme postupovat například takto: 
\begin{equation} \label{1}
  \NSPACE{\sqrt{n}} \subseteq \TIME{2^{n^{3/4}}}
\end{equation}
\begin{equation} \label{2}
  \TIME{2^{n^{3/4}}} \subsetneq \TIME{2^n}
\end{equation}
kde rovnice \ref{1} plyne z použití (D), protože $\sqrt{n} = o(n^{3/4})$.
A rovnice \ref{2} plyne z použití (G), protože
$$2^{n^{3/4}} = o(\frac{2^n}{\log 2^n}) = o(\frac{2^n}{n})$$
a toto platí, protože
$$\lim_{n \rightarrow \infty} \frac{n 2^{n^{3/4}}}{2^n} = 0$$ \\

Tedy platí $\NSPACE{\sqrt{n}} \subsetneq \TIME{2^n}$.


\section*{2)}
\emph{Porovnejte $\NSPACE{(\log n)^3}$ a $\SPACE{n}$}.\\

Postupujme následovně:
\begin{equation} \label{3}
  \NSPACE{(\log n)^3} \subseteq \SPACE{(\log n)^6}
\end{equation}
\begin{equation} \label{4}
  \SPACE{(\log n)^6} \subsetneq \SPACE{n}
\end{equation}
Kde rovnice \ref{3} vychází z (E) a rovnice \ref{4} vychází z (F), protože
$(\log n)^6 = o(n)$ a toto platí protože
$\lim_{n \rightarrow \infty} \frac{(\log n)^6}{n} = 0$.
A tedy platí $\NSPACE{(\log n)^3} \subsetneq \SPACE{n}$.


\section*{3)}
\emph{Porovnejte $\NTIME{n^3}$ a $\SPACE{n^6}$}.\\

Postupujme následovně:
\begin{equation} \label{5}
  \NTIME{n^3} \subseteq \SPACE{n^3}
\end{equation}
\begin{equation} \label{6}
  \SPACE{n^3} \subsetneq \SPACE{n^6}
\end{equation}
Kde rovnice \ref{5} vychází z (B) a rovnice \ref{6} vychází z (F), protože
$n^3 = o(n^6)$. Tedy platí $\NTIME{n^3} \subsetneq \SPACE{n^6}$.


\end{document}