\documentclass{article}

\usepackage[utf8]{inputenc}
\usepackage[czech]{babel}
\usepackage{amsmath}
\usepackage{xcolor}

\begin{document}

\title{Du 3 - verze 2}
\author{Pavel Marek}
\date{}

\maketitle

\section*{1)}
Rozhodněte, zda jazyk $S = \{\langle M_1, M_2 \rangle \mid
L(M_1) \cap L(M_2) = \emptyset\}$ je rozhodnutelný.

\section*{2.1)}
Ukažte, že $L_u \leq_m S$, kde $S=\{\langle M \rangle \mid
(\forall x \in \Sigma^*)
[x \in L(M) \Leftrightarrow x^R \in L(M)]\}$.\\

Chceme převést $\langle M, x \rangle \rightarrow \langle M' \rangle$.
Výpočet $M'$ bude vypadat následovně: \\
$M'(y \in \Sigma^*)$:
\begin{enumerate}
  \item Pokud $y = a$, tak přijmi. \emph{$a$ je předem zvolené konstantní
  slovo}.
  \item Spusť $M(x)$. Pokud odmítne, odmítni. \emph{Zde nám nevadí, pokud
  výpočet $M(x)$ nedoběhne.}
  \item Pokud $y = a^R$, tak přijmi.
  \item Odmítni.
\end{enumerate}

Všimněme si, že pokud $x \in L(M)$, tak $L(M') = \{a, a^R\}$. A pokud
$x \notin L(M)$, tak $L(M') = \{a\}$. Tím pádem je splněna podmínka
$x \in L(M) \Leftrightarrow ((\forall x \in \Sigma^*)
(x \in M' \Leftrightarrow x^R \in M'))$.


\section*{2.2)}
Ukažte, že $L_u \leq_m \overline{S}$, kde $\overline{S} = \{\langle M \rangle
\mid (\forall x \in \Sigma^*) [x \in L(M) \Leftrightarrow x^R \notin L(M)]\}$.\\

Chceme převést $\langle M, x \rangle \rightarrow \langle M' \rangle$. 
Výpočet $M'$ bude vypadat následovně: \\
$M'(y \in \Sigma^*)$:
\begin{enumerate}
  \item Pokud $y = a$, přijmi.
  \item Pokud $y = a^R$, přijmi. \textcolor{red}{První dva kroky jsou zde
  zbytečné. Prázdný jazyk je také $\overline{S}$}.
  \item Spusť $M(x)$. Pokud odmítne, přijmi.
  \item Pokud $y = b$, přijmi. \emph{$b$ je předem zvolené konstantní slovo.}
  \item Odmítni.
\end{enumerate}

Všimněme si, že pokud $x \in L(M)$, tak $L(M') = \{a, a^R, b\}$. A pokud
$x \notin L(M)$, tak $L(M') = \{a, a^R\}$. Tím pádem je splněna podmínka
$x \in L(M) \Leftrightarrow ((\forall x \in \Sigma^*)
(x \in M' \Leftrightarrow x^R \notin M'))$.


\end{document}
