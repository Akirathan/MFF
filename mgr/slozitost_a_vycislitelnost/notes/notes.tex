\documentclass{article}

\usepackage[utf8]{inputenc}
\usepackage[czech]{babel}
\usepackage{amsmath}
\usepackage{amsthm}
\usepackage{amsfonts}
\usepackage{xcolor}
\usepackage{amssymb}
\usepackage{hyperref}
\usepackage{enumitem}

\newcommand{\TIME}[1]{\mathrm{TIME}(#1)}
\newcommand{\NTIME}[1]{\mathrm{NTIME}(#1)}
\newcommand{\SPACE}[1]{\mathrm{SPACE}(#1)}
\newcommand{\NSPACE}[1]{\mathrm{NSPACE}(#1)}

\theoremstyle{plain}
\newtheorem{veta}{Věta}
\newtheorem{lemma}[veta]{Lemma}
\newtheorem{tvrz}[veta]{Tvrzení}

\theoremstyle{plain}
\newtheorem{definice}{Definice}

\theoremstyle{remark}
\newtheorem*{dusl}{Důsledek}
\newtheorem*{pozn}{Poznámka}
\newtheorem*{prikl}{Příklad}

\newenvironment{dukaz}{
  \par\medskip\noindent
  \textit{Důkaz}.
}{
\newline
\rightline{$\square$}  % nebo \SquareCastShadowBottomRight z balíčku bbding
}

\begin{document}

\begin{itemize}
  \item Jazyk $L$ je \emph{rozhodnutelný}, pokud pro něj existuje turingův
  stroj $T$ takový, že $L(T) = L$ a navíc se pro každý vstup zastaví.

  \item Jazyk $L$ je \emph{částečné rozhodnutelný}, pokud existuje turingův
  stroj $T$, že $L(T) = L$.

  \item Ne všechny jazyky jsou částečně rozhodnutelné.

  \item \textbf{Postova věta:} Jazyk $L$ je rozhodnutelný $\Leftrightarrow$
  $L$ i $\overline{L}$ jsou částečně rozhodnutelné.

  \item $A \leq_m B$ a zároveň $B$ je částečně rozhodnutelný $\Rightarrow$
  $A$ je částečně rozhodnutelný.

  \item $A$, $B$ rozhodnutelné jazyky, potom $A \leq_m B$.

  \item $A$ je částečně rozhodnutelný jazyk a $A \leq_m \overline{A}$,
  pak $A$ je rozhodnutelný jazyk.

  \item Pokud je $A$ rozhodnutelný, pak je i částečně rozhodnutelný.

  \item $HALT := \{\langle M, x \rangle \mid M(x) \downarrow \}$.
  \begin{itemize}
    \item Pokud $M(x)$ zastaví, nemusí přijmout.
  \end{itemize}

  \item $L_u := \{\langle M, x \rangle \mid x \in L(M)\}$.

  \item $HALT$ a $L_u$ jsou částečně rozhodnutelné, ale nejsou rozhodnutelné.
  \begin{itemize}
    \item $\overline{L_u}$ není částečně rozhodnutelný.
  \end{itemize}

  \item Jazyk $DIAG := \{\langle M \rangle \mid \langle M \rangle \notin L(M)\}$
  není částečně rozhodnutelný.

  \item Jazyk $EMPTY := \{\langle M \rangle \mid L(M) = \emptyset\}$ není
  částečně rozhodnutelný.

  \item Jak o jazyku dokázat, že není rozhodnutelný?
  \begin{itemize}
    \item $L$ je nerozhodnutelný (ale je částečně rozhodnutelný), pokud
    $L_u \leq_m L$.

    \item $L$ není částečně rozhodnutelný, pokud $DIAG \leq_m L$. Nebo pokud
    $EMPTY \leq_m L$.
  \end{itemize}

  \item \emph{Totálně vyčíslitelná funkce} je turingovsky vyčíslitelná
  turingovským strojem, který se zastaví pro každý vstup.
\end{itemize}

\section{Složitost}
\subsection{Vztahy mezi třídami}

\begin{itemize}
  \item \textbf{Big-O notation} - \href{https://en.wikipedia.org/wiki/Big_O_notation#Family_of_Bachmann%E2%80%93Landau_notations}{Wikipedia}
  \item $\TIME{f(n)} \subseteq \SPACE{f(n)}$. \emph{Triviálně}.
  \item (\texttt{A}) $\NTIME{f(n)} \subseteq \NSPACE{f(n)}$. \emph{Triviálně}.
  \item $\NTIME{f(n)} \subseteq \SPACE{f(n)}$. \emph{Každá větev výpočtu se vejde
  do $f(n)$}.

  \item (\texttt{B}) $\TIME{f(n)} \subseteq \NTIME{f(n)} \subseteq \SPACE{f(n)}
  \subseteq \NSPACE{f(n)}$. \emph{Vychází z předchozích}.

  \item Nedeterministický turingův stroj $M$, pracující v prostoru $f(n)$ má
  nanejvýš $2^{C_M f(n)}$ konfigurací. Konfigurace je:
  \begin{itemize}[noitemsep]
    \item Slovo na vstupní pásce
    \item Poloha hlavy na všech páskách
    \item Stav
  \end{itemize}

  \item (\texttt{C}) $f(n)$ taková, že $f(n) \geq \log_2n$, pak pro každý jazyk
  $L$ platí $L \in \NSPACE{f(n)} \Rightarrow (\exists C_L \in \mathbb{N}) (L \in
  \TIME{2^{C_L f(n)})}$.
  \begin{itemize}
    \item (\texttt{D}) $f(n) = o(g(n))$, $\NSPACE{f(n)} \subseteq
    \TIME{2^{g(n)}}$.
  \end{itemize}

  \item (\texttt{E}) \textbf{Savičova věta} $\NSPACE{f(n)} \subseteq
  \SPACE{f^2(n)}$.

  \item \textbf{Věta o prostorové deterministické hierarchii} Pro každou $f$
  prostorově konstruovatelnou existuje jazyk $A$, který je rozhodnutelný v
  prostoru $O(f(n))$, ale není rozhodnutelný v prostoru $o(f(n))$.
  \begin{itemize}
    \item (\texttt{F}) $f(n) = o(g(n))$, $g$ je prostorově konstruovatelná.
    Potom $\SPACE{f(n)} \subsetneq \SPACE{g(n)}$.
  \end{itemize}

  \item \textbf{Věta o časové deterministické hierarchii} Pro každou $f$
  časově konstruovatelnou existuje jazyk $A$, který je rozhodnutelný v čase
  $O(f(n))$, ale ne v čase $o(f(n)/\log f(n))$
  \begin{itemize}
    \item (\texttt{G}) $f(n) = o(\frac{g(n)}{\log g(n)})$, $g$ je časově
    konstruovatelná. Potom $\TIME{f(n)} \subsetneq \TIME{g(n)}$.
  \end{itemize}
\end{itemize}

\end{document}