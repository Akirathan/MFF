\documentclass{article}

\usepackage[utf8]{inputenc}
\usepackage[czech]{babel}
\usepackage{amsmath}

\begin{document}

\title{Du 2}
\author{Pavel Marek}
\date{}

\maketitle

\section*{1) $\Rightarrow$}
$f$ je algoritmicky vyčíslitelná $\Rightarrow$ 
$B = \{\langle x,y \rangle|f(x)=y\}$ je
částečně rozhodnutelná.

\paragraph{Důkaz}
Máme turingův stroj $M$, který počítá $f$. Zkonstruujeme $M'$ takový, že
$L(M')=B$: \\

$M'(x \in \Sigma^*)$:
\begin{enumerate}
  \item Pro všechny rozdělení $\langle x', y' \rangle$ z $x$:
  \begin{enumerate}
    \item Spusť $M(x')$. Jeho výstup nazvěme $y$.
    \item Pokud $y = y'$:
    \begin{enumerate}
      \item Přijmi $\langle x', y' \rangle$. \emph{Zde totiž platí, že $f(x') = y'$}.
    \end{enumerate}
  \end{enumerate}
  \item Odmítni $x$.
\end{enumerate}

K bodu 1 poznamenejme, že rozdělení vstupního slova $x$ lze enumerovat například
tak, že budeme posunovat rozdělovač $x'$ a $y'$ doprava.



\section*{1) $\Leftarrow$}
$B=\{\langle x,y \rangle|f(x)=y\}$ je částečně rozhodnutelná $\Rightarrow$ $f$
je algoritmicky vyčíslitelná.

\paragraph{Důkaz}
Máme $M$ takový, že $L(M) = B$. Zkonstruujeme $M'$ takový, že $M'$ počítá $f$:\\

$M'(x \in \Sigma^*)$:
\begin{enumerate}
  \item $y = \epsilon$.
  \item $L$ je prázdný seznam procesů. \emph{K procesům si budeme přidávat
  ještě část jejich vstupu}.
  \item Dokud žádný proces v $L$ nepřijal:
  \begin{enumerate}
    \item Přidej proces $\langle x,y \rangle \in B$ společně s $y$ do $L$.
    \emph{Tento proces lze jednoduše implementovat tak, že spustíme
    $M(\langle x,y \rangle )$}.
    \item $y =$ \texttt{next}$(y)$.
  \end{enumerate}
  \item Vypiš $y$. \emph{Zde už víme, že nějaký proces z $L$ přijal a $y$ je
  část jeho vstupu. Navíc máme jistotu, že $\langle x,y \rangle \in B$ a $f(x) = y$}.
\end{enumerate}


\section*{2.1)}
\texttt{HASPRIME} $= \{\langle M \rangle |
\exists p(p$ prvočíslo a $\langle p \rangle \in L(M)\}$.
Sestrojíme $M'$ takový, že $L(M') =$ \texttt{HASPRIME} :\\

$M'(x \in \Sigma^*)$:
\begin{enumerate}
  \item Ověř, že $x$ je kód nějakého turingova stroje a označ
  $\langle M \rangle = x$.
  \item Pokud není, odmítni.
  \item Pro všechny $p$ prvočísla:
  \begin{enumerate}
    \item Spusť $M(\langle p \rangle)$. \emph{Nevadí, když výpočet nedoběhne.}
    \item Pokud přijme, přijmi.
    \item Jinak odmítni.
  \end{enumerate}
\end{enumerate}

Ještě pro krok 3 poznamenejme, že prvočísla jsou rozhodnutelný jazyk.
Tedy lze pro ně sestrojit enumerátor, který je vypisuje v lexikografickém pořadí.

\section*{2.2)}
\texttt{PSE} $= \{\langle M,q \rangle | \exists x ( M$ použije při výpočtu $x$ stav $q)$.
Sestrojme turingův stroj $M'$ tak, že $L(M') = $ \texttt{PSE}: \\

$M'(x \in \Sigma^*)$:
\begin{enumerate}
  \item Ověř, že $x$ je syntakticky $\langle M,q \rangle$ a přiřaď
  $\langle M,q \rangle = x$.
  \item Modifikujeme $M$:
  \begin{enumerate}
    \item Změníme $q$ na \textbf{jediný} přijímající stav.
    \item Smažeme všechny přechody vedoucí ze stavu $q$.
  \end{enumerate}
  \item Pro všechny $y \in \Sigma^*$: \emph{Zvolíme libovolné uspořádání a
  enumerujeme všechny slova v jazyce}.
  \begin{enumerate}
    \item Spustíme $M(y)$. \emph{Nevadí, když výpočet nedoběhne}.
    \item Jestli přijme, přijmi.
    \item Jinak odmítni.
  \end{enumerate}
\end{enumerate}

\end{document}